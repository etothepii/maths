\documentclass[a4paper,10pt]{article}
\usepackage{parskip}
\usepackage{amsmath}
\usepackage{bbold}
\title{The Fresneda-Quiroga Conjecture}
\author{James P. Robinson}
\begin{document}
\maketitle
%\begin{abstract}
%\noindent
%I owe a debt to my collegue Roman Fresneda-Quiroga who introduced me to this question. It has given me a great deal of pleasure over the last couple of days and I have very much enjoyed proving it, more so as on two days ago I had convinced myself that the fact that there is no solution in the case $n=19$ with $(a,b)=1$ meant that I had a counter example. A moments hesitation in LIN-Q Pad would have allowed me to see that $a = 54\textrm{, }b=10$ is a valid solution for the case $n=19$.
%\end{abstract}
Consider the integer equation $$\frac{a^2+b}{b^2+a}=n$$
\begin{section}{$n = 1$}
This case is trivial simply concider any solution with $a = b$
\end{section}
\begin{section}{$n$ is not a perfect square}
If $(a,b)=c$ then $\exists$ $a',b'$ such that $(a',b')=1$ with $a=a'c$ and $b=b'c$. Thus we are required only to find a solution such that $$\frac{(a'c)^2 + b'c}{(b'c)^2+a'c} = \frac{ca'^2+b'}{cb'^2+a'} = n$$ 
Which by renaming means we need only find  $$\frac{ca^2+b}{cb^2+a}=n$$ With $(a,b) = 1$ and $c\in\mathbb{N}$
Thus 
\begin{eqnarray*}
  \frac{ca^2+b}{cb^2+a}&=&n\\
  ca^2+b&=&n(cb^2+a)\\
  c(a^2-nb^2)&=&an-b\\
  c&=&\frac{an-b}{a^2-nb^2}
\end{eqnarray*}
Thus by Pell's equation we can choose $a$ and $b$ from amongst the continued fractions of $\sqrt{n}$ to yield $$a^2-nb^2=1$$ and thus one can calculate $$c=an-b$$ 
\end{section}
\begin{section}{$n=m^2$}
From the previous section we have that $$c(a^2-m^2b^2)=am^2-b$$
and thus $$c(a-mb)=\frac{am^2-b}{a+mb}=m^2-b\frac{m^3+1}{a+mb}$$
and $$c(a+mb)=\frac{am^2-b}{a-mb}=m^2+b\frac{m^3-1}{a-mb}$$
\begin{subsection}{$m$ is even}
  If $m$ is even we can simply let $a-mb=1$ and $a+mb=m^3+1$ which yields
$$a=\frac{m^3}2+1\textrm{, }b=\frac{m^2}2\textrm{ and }c = \frac{m^2}2$$
\end{subsection}
\begin{subsection}{$m$ is odd}
  If m is odd the above approach will not yield integer solutions. However if we try sticking with $a-mb = 1$ and change to $a+mb=m^2-m+1$ then we get
  $$a=\frac{m^2-m+2}2\textrm{, }b=\frac{m-1}2\textrm{ and }c=\frac{m^2+1}2$$
\end{subsection}
\end{section}
\end{document}
